\documentclass[../main.tex]{subfile}

\begin{document}

\section{Introduction} \label{sec:introduction}
    Fruit selection is a highly important economic activity. The traditional method of selection is made by human labor. During postharvest, some people are responsible for visually inspect fruits and classify them according to their color, weight, maturity, diseases, deformations, among other attributes.
    
    An alternative to the traditional method is to automatize the process with the help of fruit selecting machines. They can have several conveyor belts available for carrying the fruit towards their sensors. On them, data is obtained, and through software, the destination of each fruit is decided. Upon determining their destinations, the software acts on the conveyor belts to ensure that the fruits arrive correctly.
    
    Automizing this process comes with three benefits:
    \begin{itemize}
        \item Error reduction. Given its repetitive nature, humans are more prone to errors when compared to machines;
        
        \item Cost-effectiveness. Classifying fruits fast and correctly demands a lot of training. Training people takes time, needs to be done for every new employee, and it is not certain that newly trained employees will perform the same way as more experienced employees. On the other hand, to replicate a selecting machine, building a new one and loading the same software is enough;
        
        \item Higher rate of selection. There is a maximum rate of how many fruits a human can classify per second. A machine can easily go beyond that limit.
    \end{itemize}
    
    A more intuitive solution is to use traditional cameras to photograph fruits while the software decides those fruits' destinations based on those images. Under this perspective, the challenges of building those fruit selecting machines involve computer vision.
    
    Machine learning suggests the usage of deep neural networks to approximate functions of all types of complexities. The functions are built based on the union of simple computational units, and they are capable of recognizing relevant patterns to complete a task.
    
    Therefore, the usage of machine learning is already established and offers promising results regarding the problem of detecting external defects on fruit. Pandey et al. \cite{I-item1} revise the literature on automatic fruit selection. The goal of this paper is to develop a computer vision algorithm as good as humans in identifying external defects on fruit with the help of deep learning.
    
\end{document}