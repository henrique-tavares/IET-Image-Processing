\documentclass[../main.tex]{subfile}

\begin{document}

\section{Conclusions \& Future Scope} \label{sec:conclusions-&-future-scope}

In this work, we presented some solutions related to the problem of classifying fruit according to their quality. The solutions presented were: a CNN with an RGB or grayscale image; and a DNN with GLCM textural properties. A Gabor filter was applied to all of them.

The DNN with GLCM textural properties outperformed the other solutions for all the six fruits, with significantly better results in most cases. This means that texture is a key factor when analyzing fruits based on their appearance.

\subfile{../tables/related-work}

This solution proved to be effective, as even with a small dataset, i.e. 120 samples, still managed to achieve great accuracies as stated in the previous section, and also when comparing the results of previous works as seen in Table \ref{tab:related-work}. Further experimentations can be done by using larger datasets and testing even more possible solutions.


\end{document}