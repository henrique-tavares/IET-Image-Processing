\documentclass[../main.tex]{subfile}

\begin{document}

\section{Conclusions \& Future Scope} \label{sec:conclusions-&-future-scope}

In this work, we presented some solutions related to the problem of classifying fruit according to their quality. The solutions presented included: a CNN with an RGB or grayscale image; and a DNN with GLCM textural properties. A Gabor filter was applied to all of them.

The DNN with GLCM textural properties outperformed the other solutions for all the six fruits, with significantly better results in most cases. This means that texture is a key factor when analyzing fruits based on their appearance.

\subfile{../tables/related-work}

This solution proved to be effective, as even with a small dataset, i.e. 120 samples, still managed to achieve great accuracies as stated in the previous section, and also when comparing the results of previous works as seen in Table \ref{tab:related-work}.

In the future, this fruit quality detection proposal should be extended to other tropical fruits with economic potential. Furthermore, this proposal should be compared with other automated techniques and some new parameters or features can be added. The proposed model is capable of detecting the quality of a fruit at a time and this can be expanded in to detect multiple fruits of different types at the same time. The efficiency in detect the precision level of fruit quality can be increased and time consuming can be reduced to a short period of time. The classifier can also be controlled via cell phone application.

\end{document}