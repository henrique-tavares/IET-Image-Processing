\documentclass[../main.tex]{subfile}

\begin{document}

\section{Related Work} \label{sec:related-work}
    Image processing and computer vision techniques have been growing important for the fruit industry over the years, especially when applied to quality inspection tasks and defect classification.
    
    It is presented in \cite{II-item1} an effort in recognizing defects on apples using a machine vision system based on three colors cameras. The apple image is segmented from its background by using a multiple threshold method, then, defect detection and counting are made upon the apple image.
    
    Various techniques are analyzed in \cite{II-item2} for automatic fruit inspection with the help of computer vision. The analyzed fruits are apples and citric ones. The authors revised possible algorithms for defects detection on those fruits. The solutions are different and well suited for each type of fruit.
    
    The proposed method in \cite{II-item3} uses pre-processing, segmentation, border detection, and features extraction to classify a fruit as either defective or fresh.
    
    Fruit peel color is an important factor in identifying defects on fruits in \cite{II-item4}. The procedures adopted are background segmentation, identification of spots for a quality calculation, and a classification process.
    
    A fruit defect detection in \cite{II-item5} is made by using some local features such as energy, homogeneity, contrast, and correlation, all from segmented and filtered images.
    
    The purpose of the work made in \cite{II-item6} is to identify whether mangos are mature or not. The fruit images are obtained and then the fruits are harvested if mature. From the images, after its background is removed features such as shape, color, and size are extracted. The result is a classification of whether the fruit is mature or not.
    
    Another fruit external defects classification is made in \cite{II-item7}. The images are collected, segmented, and then a color histogram, global correlation, and local binary pattern are calculated in order to feed an SVM classifier.
    
    The work in \cite{II-item8} consists of analyzing orange images and classifying them. Features such as color, texture, and shape are extracted, and then an RBPNN (Radial Basis Probabilistic Neural Network) is used to perform the classification.
    
    Evaluated articles use some kind of pre-processing (e.g., background removal and segmentation). Direct features (color, texture, shape, size) or secondary features (color histogram, global correlation, energy, homogeneity, contrast, and correlation), or both are extracted. Aside from the pre-processing, they also use different classification techniques: algorithms based on thresholds, Support Vector Machines, or neural networks.
\end{document}